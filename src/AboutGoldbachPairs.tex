\documentclass[12pt,letterpaper,oneside,reqno]{amsart}
\usepackage{amsfonts}
\usepackage{amsmath}
\usepackage{amssymb}
\usepackage{amsthm}
\usepackage{float}
\usepackage{mathrsfs}
\usepackage{colonequals}
\usepackage[font=small,labelfont=bf]{caption}
\usepackage[left=1in,right=1in,bottom=1in,top=1in]{geometry}
\usepackage[pdfpagelabels,hyperindex,colorlinks=true,linkcolor=blue,urlcolor=magenta,citecolor=green]{hyperref}
\usepackage{graphicx}
\linespread{1.7}
\emergencystretch=1em
\usepackage{array}
\usepackage{etoolbox}
\apptocmd{\sloppy}{\hbadness 10000\relax}{}{}
\raggedbottom

\newcommand \anglePower [2]{\langle #1 \rangle \sp{#2}}
\newcommand \bernoulli [2][B] {{#1}\sb{#2}}
\newcommand \curvePower [2]{\{#1\}\sp{#2}}
\newcommand \coeffA [3][A] {{\mathbf{#1}} \sb{#2,#3}}
\newcommand \polynomialP [4][P]{{\mathbf{#1}}\sp{#2} \sb{#3}(#4)}

% ordinary derivatives
\newcommand \derivative [2] {\frac{d}{d #2} #1}                              % 1 - function; 2 - variable;
\newcommand \pderivative [2] {\frac{\partial #1}{\partial #2}}               % 1 - function; 2 - variable;
\newcommand \qderivative [1] {D_{q} #1}                                      % 1 - function
\newcommand \nqderivative [1] {D_{n,q} #1}                                   % 1 - function
\newcommand \qpowerDerivative [1] {\mathcal{D}_q #1}                         % 1 - function;
\newcommand \finiteDifference [1] {\Delta #1}                                % 1 - function;
\newcommand \pTsDerivative [2] {\frac{\partial #1}{\Delta #2}}               % 1 - function; 2 - variable;

% high order derivatives
\newcommand \derivativeHO [3] {\frac{d^{#3}}{d {#2}^{#3}} #1}                % 1 - function; 2 - variable; 3 - order
\newcommand \pderivativeHO [3]{\frac{\partial^{#3}}{\partial {#2}^{#3}} #1}
\newcommand \qderivativeHO [2] {D_{q}^{#2} #1}                               % 1 - function; 2 - order
\newcommand \qpowerDerivativeHO [2] {\mathcal{D}_{q}^{#2} #1}                % 1 - function; 2 - order
\newcommand \finiteDifferenceHO [2] {\Delta^{#2} #1}                         % 1 - function; 2 - order
\newcommand \pTsDerivativeHO [3] {\frac{\partial^{#3}}{\Delta {#2}^{#3}} #1} % 1 - function; 2 - variable;

% central factorials and related symbols
\newcommand \centralFactorial [2] {#1^{[#2]}}
\newcommand \fallingFactorial [2] {\left(#1 \right)^{\underline{#2}}}
\newcommand{\stirlingii}{\genfrac{\{}{\}}{0pt}{}}
\newcommand{\eulerianNumber}{\genfrac{\langle}{\rangle}{0pt}{}}

% for llceil coeffcient
\newcommand{\nobarfrac}{\genfrac{}{}{0pt}{}}
\def\llceil{\left\lceil\kern-3.5pt\left\lceil}
\def\rrfloor{\right\rfloor\kern-3.5pt\right\rfloor}
\newcommand \llceilCoefficient [3] {\llceil \nobarfrac{#1}{#2} \rrfloor_{#3}}

% ~~~ Rascal numbers ~~~
%\newcommand \rascalNumber [3] {\binom{#1}{#2}_{#3}}
%\newcommand \north[0] {\mathbf{North}}
%\newcommand \south[0] {\mathbf{South}}
%\newcommand \west[0] {\mathbf{West}}
%\newcommand \east[0] {\mathbf{East}}

% ~~~~ 1-q pascal notation~~~~
%\newcommand{\genstirlingI}[3]{%
%    \genfrac{[}{]}{0pt}{#1}{#2}{#3}%
%}
%\newcommand{\genstirlingII}[3]{%
%    \genfrac{\{}{\}}{0pt}{#1}{#2}{#3}%
%}
%\newcommand{\oneQBinomial}[3]{\genstirlingI{}{#1}{#2}^{#3}}

% free foot note
\let\svthefootnote\thefootnote
\newcommand\freefootnote[1]{%
    \let\thefootnote\relax%
    \footnotetext{#1}%
    \let\thefootnote\svthefootnote%
}


\newtheorem{theorem}{Theorem}[section]
\newtheorem{corollary}[theorem]{Corollary}
\newtheorem{lemma}[theorem]{Lemma}
\newtheorem{example}[theorem]{Example}
\newtheorem{conjecture}[theorem]{Conjecture}
\newtheorem{definition}[theorem]{Definition}

\numberwithin{equation}{section}

\title[LaTeX Template for Github]
{LaTeX Template for Github}
\author[Petro Kolosov]{Petro Kolosov}
\date{\today}
%\address{Software Developer, DevOps Engineer}
%\email{kolosovp94@gmail.com}
%\urladdr{https://kolosovpetro.github.io}
%\keywords{
%    Binomial theorem,
%    Binomial coefficients,
%    Faulhaber's formula,
%    Polynomials,
%    Pascal's triangle
%    Finite differences,
%    Interpolation,
%    Polynomial identities
%}
%\subjclass[2010]{26E70, 05A30}
%\hypersetup{
%    pdftitle={LaTeX Template for Github},
%    pdfsubject={
%        Polynomials,
%        Finite differences,
%        Interpolation,
%        Approximation,
%        Polynomial identities,
%        Power sums,
%        Binomial theorem,
%        Power function,
%        Binomial coefficients,
%        Bernoulli numbers,
%        Pascal's triangle,
%        Faulhaber's formula,
%        Derivatives,
%        Differential calculus,
%        Partial differential equations,
%        OEIS,
%        Bernoulli polynomials,
%        Combinatorics,
%        Discrete convolution,
%        Dynamic systems,
%        Time scales
%    },
%    pdfauthor={Petro Kolosov},
%    pdfkeywords={
%        Polynomials,
%        Finite differences,
%        Interpolation,
%        Approximation,
%        Polynomial identities,
%        Power sums,
%        Binomial theorem,
%        Power function,
%        Binomial coefficients,
%        Bernoulli numbers,
%        Pascal's triangle,
%        Faulhaber's formula,
%        Derivatives,
%        Differential calculus,
%        Partial differential equations,
%        OEIS,
%        Bernoulli polynomials,
%        Combinatorics,
%        Discrete convolution,
%        Dynamic systems,
%        Time scales
%    }
%}
\begin{document}
%    \begin{abstract}
%        Goldbach's Conjecture assumes that every even integer $2N \geq 4$ can be written as the sum of two primes
$2N = p_i + p_j$, where $(p_i, p_j)$ is called a Goldbach pair.
The minimal Goldbach pair is a pair $(p_i, p_j)$ such that $p_i$ is minimal and $p_j = 2N - p_i$ is also a prime.
We define a function $F_{2N}(P)$ that counts the occurrences of $p_i = P$ in a set of minimal Goldbach pairs
up to $2N$, where $P$ is a fixed prime number.
In particular, the function $F_{2N}(P)$ provides the following identities in terms of prime counting $\pi(2N)$ and
twin-prime counting $\pi_2(2N), \; \pi_4(2N)$
\begin{align*}
    \pi(2N)   &= F_{2N+3}(3) + 1 \\
    \pi_2(2N) &= F_{2N+3}(3) - F_{2N+5}(5) \\
    \pi_4(2N) &= F_{2N}(5) - F_{2N}(7)
\end{align*}

%    \end{abstract}

    \maketitle

%    \tableofcontents

%    \freefootnote{Sources: \url{https://github.com/kolosovpetro/github-latex-template}}

    \section{Definitions} \label{sec:introduction}
    \usepackage{amsmath}Goldbach conjecture states that every even integer $N$ greater than 2 is a sum of two primes.

$$N = p_i + p_j$$

where $(p_i, p_j)$ is called Goldbach pair.

Goldbach pair is not unique for some even integers, meaning that there can be multiple goldbach pairs for even integer $N$.

For example: $10=3+7$ and $10=5+5$ and $10=7+3$ where goldbach pairs are $(3,7), (5,5), (7,3)$.

Minimal goldbach pair is the pair having minimal $p_i$ across all goldbach pairs for even integer $N$.

For even integer $10$ we have three pairs $(3,7), (5,5), (7,3)$ while the minimal is $(3,7)$ because
$3$ is the minimal value in the $p_i$ set: $3, 5, 7$

\subsection{Function F}\label{subsec:function-f}
$F_{n}(P)$ counts the number of minimal goldbach pairs $(p_i, p_j)$ such that $p_i=P$ within the interval $6 \leq k \leq n$,
where $P$ is a prime.
For example, consider the case $F_{20}(3)$.
First, we get a set of minimal goldbach pairs within the range $6 \leq k \leq 20$, that is

\begin{align*}
    6 &= 3 + 3, \\
    8 &= 3 + 5, \\
    10 &= 3 + 7, \\
    12 &= 5 + 7, \\
    14 &= 3 + 11, \\
    16 &= 3 + 13, \\
    18 &= 5 + 13, \\
    20 &= 3 + 17
\end{align*}
Therefore, the function $F_{20}(3)$ gives $6$ because there are only six minimal goldbach pairs $(p_i, p_j)$ such that
$p_i=3$, that are:
\begin{align*}
    6 &= 3 + 3, \\
    8 &= 3 + 5, \\
    10 &= 3 + 7, \\
    14 &= 3 + 11, \\
    16 &= 3 + 13, \\
    20 &= 3 + 17
\end{align*}
What is also interesting to notice, is that $p_j$ in the example above produces the consecutive sequence of prime numbers,
$p_j = 3, 5, 7, 11, 13, 17 \dots$
In general, for every $n \geq 2$
\begin{equation*}
    \pi (n) = F_{n+3}(3) +1
\end{equation*}



    \bibliographystyle{unsrt}
    \bibliography{AboutGoldbachPairs}
%    \noindent \textbf{Version:} \texttt{Local-0.1.0}


\end{document}
