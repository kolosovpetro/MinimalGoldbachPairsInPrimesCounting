This manuscript provides a comprehensive review of the work~\cite{yamagishi2025goldbach} done by Michel Yamagishi.
The Goldbach conjecture asserts that every even integer $2N \geq 4$ is a sum of two primes
\begin{align*}
    2N = p_i + p_j
\end{align*}
where $(p_i, p_j)$ is called a Goldbach pair.

A Goldbach pair is not unique for even integers greater than 6,
meaning that there can be multiple Goldbach pairs for even integer $2N \geq 8$.
For example: $10=3+7$ and $10=5+5$ and $10=7+3$ where goldbach pairs are $(3,7), \; (5,5), \; (7,3)$.

The minimal Goldbach pair is the pair having minimal $p_i$ among all goldbach pairs for even integer $2N$.
For the even integer $10$, we have three pairs: $(3,7), \; (5,5), \; (7,3)$, while the minimal is $(3,7)$ because
$3$ is the smallest among all $p_i$ values: $3, 5, 7$.


Consider the following minimal goldbach pairs for even integer $2k$ within the range $6 \leq 2k \leq 50$
\begin{align*}
    6  &= 3 + 3,   & 8  &= 3 + 5,   \\
    10 &= 3 + 7,   & 12 &= 5 + 7,   \\
    14 &= 3 + 11,  & 16 &= 3 + 13,  \\
    18 &= 5 + 13,  & 20 &= 3 + 17,  \\
    22 &= 3 + 19,  & 24 &= 5 + 19,  \\
    26 &= 3 + 23,  & 28 &= 5 + 23,  \\
    30 &= 7 + 23,  & 32 &= 3 + 29,  \\
    34 &= 3 + 31,  & 36 &= 5 + 31,  \\
    38 &= 7 + 31,  & 40 &= 3 + 37,  \\
    42 &= 5 + 37,  & 44 &= 3 + 41,  \\
    46 &= 3 + 43,  & 48 &= 5 + 43,  \\
    50 &= 3 + 47
\end{align*}
We can notice that minimal Goldbach pairs having $p_i=3$ produce a sequence of odd prime numbers
$p_j = 3, 5, 7, 11, 13, 17 \dots$ which is quite remarkable:
\begin{align*}
    6  &= 3 + 3,   & 8  &= 3 + 5,   \\
    10 &= 3 + 7,   & 14 &= 3 + 11,  \\
    16 &= 3 + 13,  & 20 &= 3 + 17,  \\
    22 &= 3 + 19,  & 26 &= 3 + 23,  \\
    32 &= 3 + 29,  & 34 &= 3 + 31,  \\
    40 &= 3 + 37,  & 44 &= 3 + 41,  \\
    46 &= 3 + 43,  & 50 &= 3 + 47
\end{align*}
Another interesting observation is that by selecting the pairs with minimal $p_i=5$ yields
the sequence of primes $p_j$ such that $p_j+2$ is not prime
\begin{align*}
    12 &= 5 + 7,   & 18 &= 5 + 13,  \\
    24 &= 5 + 19,  & 28 &= 5 + 23,  \\
    36 &= 5 + 31,  & 42 &= 5 + 37,  \\
    48 &= 5 + 43.
\end{align*}
To formalize and clarify our discussion, we define a few functions.
Let $G_{\min} (2N)$ be a function that returns a set of minimal Goldbach pairs $(p_i, p_j)$ having $\min p_i$
over the range $6 \leq 2k \leq 2N$
\begin{align*}
    G_{\min} (2N) = \{(p_i, p_j) \mid p_i + p_j = 2k \mid 6 \leq 2k \leq 2N \mid \min p_i \}.
\end{align*}

For example,
\begin{align*}
    G_{\min}(20) = \{
    (3,3),\;
    (3,5),\;
    (3,7),\;
    (5,7),\;
    (3,11),\;
    (3,13),\;
    (5,13),\;
    (3,17)
    \}
\end{align*}
Let $W_{2N}(P)$ be a function that returns the set of elements $p_j$ from $G_{\min} (2N)$ having $p_i=P$
\begin{align*}
    W_{2N}(P) = \{p_j \mid (p_i, p_j) \in G_{\min} (2N) \; \mathrm{and} \; p_i = P \}
\end{align*}
Then sequence of odd prime numbers~\cite{oeis:A065091} is given by $W_{2N}(3)$
\begin{align*}
    \{ 3, \; 5, \; 7, \; 11, \; \dots, \; p \leq 2N - 3\} = W_{2N}(3)
\end{align*}
Now we can easily count the number of primes within the interval $6 \leq 2k \leq 2N$ because $\pi(2N)$ equals to
the total number of elements inside the set $W_{2N}(3)$
\begin{align*}
    \pi(2N) = F_{2N}(3) + 1
\end{align*}
where $F_{2N}(3)$ is the function that counts the number of elements inside the set $W_{2N}(3)$,
in general $F_{2N}(P) = |W_{2N}(3)|$.

Taking $P=5$ in $W_{2N}(P)$ yields a sequence of primes $p_j$ such that $p_j+2$ is not a prime~\cite{oeis:A049591}
\begin{align*}
    W_{2N}(5) = \{ 7, \; 13, \; 19, \; 23, \; 31, \; 37, \; 43, \; 47, \; 53, \; \dots, \;  p \leq 2N - 5 \}
\end{align*}
Which implies that the number of twin primes in range $6 \leq 2k \leq 2N$ can be expressed in terms of $F_{2N}(P)$
\begin{align*}
    \pi_2 (2N) = F_{2N}(3) - F_{2N}(5)
\end{align*}
where $2N=10^k+2, \; k=1,2,3,4,\dots$.
For example,
\begin{align*}
    \pi_2 (12) &= F_{12}(3) - F_{12}(5) = 2 \\
    \pi_2 (102) &= F_{102}(3) - F_{102}(5) = 8 \\
    \pi_2 (1002) &= F_{1002}(3) - F_{1002}(5) = 35 \\
    \pi_2 (10002) &= F_{10002}(3) - F_{10002}(5) = 205 \\
    \pi_2 (100002) &= F_{100002}(3) - F_{100002}(5) = 1224 \\
    \pi_2 (1000002) &= F_{1000002}(3) - F_{1000002}(5) = 8169
\end{align*}
Which matches the sequence~\cite{oeis_A007508}.

Having $P=7$ function $W_{2N}(P)$ yields the sequence of primes such that $p_j-p_i \geq 6$
where $p_j$ is the next prime after $p_i$, see~\cite{oeis:A124582}
\begin{align*}
    W_{2N}(7) = \{ 23, \; 31, \;  47, \; 53, \; 61, \;  73, \; 83, \; 89, \; 113, \; \dots, \; p \leq 2N-7 \}
\end{align*}
This allows us to count the primes with next-prime gap at least 6: $\delta_6(2N) = F_{2N}(7)$.
