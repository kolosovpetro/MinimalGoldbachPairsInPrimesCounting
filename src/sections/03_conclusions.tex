Assuming Goldbach's Conjecture holds, we introduced a framework based on minimal
Goldbach pairs to derive expressions for key prime-related functions.
Specifically, we defined the function $F_{2N}(P)$ that counts occurrences of primes $p_j$ in minimal Goldbach
pairs $(p_i, p_j)$ where $p_i = P$.
Using this framework, we obtained

\begin{itemize}
    \item The prime-counting function: $\pi(2N) = F_{2N+3}(3) + 1$
    \item The twin-prime $p_2$ counting function: $\pi_2(2N) = F_{2N+3}(3) - F_{2N+5}(5)$
    \item The twin-prime $p_4$ counting function: $\pi_4 (2N) = F_{2N}(5) - F_{2N}(7)$
    \item The count of primes with next-prime gap at least 6: $\delta_6(2N) = F_{2N+7}(7)$
\end{itemize}

These identities establish a novel connection between Goldbach partitions and classical prime number theory.
Computational examples confirm alignment with known integer sequences, reinforcing the potential
of this approach for analytical and numerical exploration of prime distributions.
All the results validated up to $N=10^8$ with source code available on
GitHub~\cite{kolosovpetro2025minimalgoldbachpairs}.
