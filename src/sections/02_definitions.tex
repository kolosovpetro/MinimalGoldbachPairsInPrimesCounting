\usepackage{amsmath}Goldbach conjecture states that every even integer $N$ greater than 2 is a sum of two primes.

$$N = p_i + p_j$$

where $(p_i, p_j)$ is called Goldbach pair.

Goldbach pair is not unique for some even integers, meaning that there can be multiple goldbach pairs for even integer $N$.

For example: $10=3+7$ and $10=5+5$ and $10=7+3$ where goldbach pairs are $(3,7), (5,5), (7,3)$.

Minimal goldbach pair is the pair having minimal $p_i$ across all goldbach pairs for even integer $N$.

For even integer $10$ we have three pairs $(3,7), (5,5), (7,3)$ while the minimal is $(3,7)$ because
$3$ is the minimal value in the $p_i$ set: $3, 5, 7$

\subsection{Function F}\label{subsec:function-f}
$F_{n}(P)$ counts the number of minimal goldbach pairs $(p_i, p_j)$ such that $p_i=P$ within the interval $6 \leq k \leq n$,
where $P$ is a prime.
For example, consider the case $F_{20}(3)$.
First, we get a set of minimal goldbach pairs within the range $6 \leq k \leq 20$, that is

\begin{align*}
    6 &= 3 + 3, \\
    8 &= 3 + 5, \\
    10 &= 3 + 7, \\
    12 &= 5 + 7, \\
    14 &= 3 + 11, \\
    16 &= 3 + 13, \\
    18 &= 5 + 13, \\
    20 &= 3 + 17
\end{align*}
Therefore, the function $F_{20}(3)$ gives $6$ because there are only six minimal goldbach pairs $(p_i, p_j)$ such that
$p_i=3$, that are:
\begin{align*}
    6 &= 3 + 3, \\
    8 &= 3 + 5, \\
    10 &= 3 + 7, \\
    14 &= 3 + 11, \\
    16 &= 3 + 13, \\
    20 &= 3 + 17
\end{align*}
What is also interesting to notice, is that $p_j$ in the example above produces the consecutive sequence of prime numbers,
$p_j = 3, 5, 7, 11, 13, 17 \dots$
In general, for every $n \geq 2$
\begin{equation*}
    \pi (n) = F_{n+3}(3) +1
\end{equation*}
