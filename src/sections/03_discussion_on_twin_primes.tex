Twin prime conjecture asserts that there are infinitely many primes $p$ such that $p+2$ is also a prime.
Previously, we have established the following relation in terms of the minimal Goldbach pairs counting function
$W_{2N}(P)$.
The set of twin primes is the set $W_{2N}(3)$ excluding the set $W_{2N}(5)$
\begin{align*}
    \mathtt{Lesser \; twin \; primes} &= \{ 3, \; 5, \; 11, \; 17, \; 29, \; 41, \; 59, \; 71, \; 101, \; 107, \; 137, \; \dots, \;  p \leq 2N - 3 \} \\
    &= W_{2N}(3) \setminus W_{2N}(5)
\end{align*}
where $W_{2N}(3)$ gives the set of odd primes $p$, and $W_{2N}(5)$ gives the set of odd primes $p$ such that $p+2$
is not a prime.
Thus, the number of twin primes up to integer $2N$ is
\begin{align*}
    \pi_2 (2N) = F_{2N+3}(3) - F_{2N+5}(5)
\end{align*}
where $F_{2N}(P)$ counts the elements inside the set $W_{2N}(3)$.
Therefore, the twin prime conjecture is identical to the assertion that
\begin{align*}
    \pi_2 (2N) = F_{2N+3}(3) - F_{2N+5}(5) = \infty, \quad \quad 2N\to\infty
\end{align*}
Thanks to Euclid, we know that there are infinitely many primes, thus
there are infinitely many minimal Goldbach pairs $(p_i, p_j) = (3, p_j)$ such that $2N=p_i+p_j$,
hence
\begin{align*}
    \lim_{2N\to\infty} F_{2N+3}(3) = \infty
\end{align*}
Consider the set $W_{2N}(5)$ which gives the sequence of odd primes $p=6n+1$ such that $p+2$ is not a prime.
Dirichlet's theorem shows that there are infinitely many primes of the form
\begin{align*}
    p = 6n+1
\end{align*}
Therefore, there are infinitely many primes $p$ such that $p+2$ is not a prime
\begin{align*}
    p+2 = 6n+3 = 3(2n+1)
\end{align*}
Consequently,
\begin{align*}
    \lim_{2N\to\infty} F_{2N+5}(5) = \infty
\end{align*}
Primes $p=6n+1$ are such that either
\begin{itemize}
    \item $p+2$ is composite
    \item $p-2$ is composite or lesser twin prime
\end{itemize}
Therefore, the function $W_{2N}(5)$ generates the set of primes of the form $p=6n+1$.
By excluding the set $W_{2N}(5)$ from the set of odd prime numbers $p$,
we obtain the sequence of lesser twin primes $p_{\ell}$
\begin{align*}
    \mathtt{Lesser \; twin \; primes} &= \{ 3, \; 5, \; 11, \; 17, \; 29, \; 41, \; 59, \; 71, \; 101, \; 107, \; 137, \; \dots, \;  p_{\ell} \leq 2N - 3 \} \\
    &= W_{2N}(3) \setminus W_{2N}(5)
\end{align*}
We know that for every $N\geq 3$
\begin{align*}
    \pi_2 (2N) = F_{2N+3}(3) - F_{2N+5}(5)
\end{align*}
Since $F_{2N+3}(3) \neq 0$ with $N\geq 2$
\begin{align*}
    \lim_{2N\to\infty} \frac{\pi_2(2N)}{F_{2N+3}(3)} = 1 - \lim_{2N\to\infty} \frac{F_{2N+5}(5)}{F_{2N+3}(3)}
\end{align*}
If $\pi_2(2N)$ is finite, then
\begin{align*}
    \lim_{2N\to\infty} \frac{\pi_2(2N)}{F_{2N+3}(3)} = 0
\end{align*}
Which implies
\begin{align*}
    \lim_{2N\to\infty} \frac{F_{2N+5}(5)}{F_{2N+3}(3)} = 1
\end{align*}
By setting $2N = 10^6$, we have
\begin{align*}
    \frac{F_{10^6+5}(5)}{F_{10^6+3}(3)} = \frac{78497}{70328} = 0.895932
\end{align*}
