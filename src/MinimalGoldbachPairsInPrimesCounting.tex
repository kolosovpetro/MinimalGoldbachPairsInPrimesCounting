\documentclass[12pt,letterpaper,oneside,reqno]{amsart}
\usepackage{amsfonts}
\usepackage{amsmath}
\usepackage{amssymb}
\usepackage{amsthm}
\usepackage{float}
\usepackage{mathrsfs}
\usepackage{colonequals}
\usepackage[font=small,labelfont=bf]{caption}
\usepackage[left=1in,right=1in,bottom=1in,top=1in]{geometry}
\usepackage[pdfpagelabels,hyperindex,colorlinks=true,linkcolor=blue,urlcolor=magenta,citecolor=green]{hyperref}
\usepackage{graphicx}
\linespread{1.7}
\emergencystretch=1em
\usepackage{array}
\usepackage{etoolbox}
\apptocmd{\sloppy}{\hbadness 10000\relax}{}{}
\raggedbottom

% free foot note
\let\svthefootnote\thefootnote
\newcommand\freefootnote[1]{%
    \let\thefootnote\relax%
    \footnotetext{#1}%
    \let\thefootnote\svthefootnote%
}


\newtheorem{theorem}{Theorem}[section]
\newtheorem{corollary}[theorem]{Corollary}
\newtheorem{lemma}[theorem]{Lemma}
\newtheorem{example}[theorem]{Example}
\newtheorem{conjecture}[theorem]{Conjecture}
\newtheorem{definition}[theorem]{Definition}

\numberwithin{equation}{section}

\title[Counting primes and twin-primes using minimal Goldbach pairs]
{Counting primes and twin-primes using minimal Goldbach pairs}
\author[Petro Kolosov]{Petro Kolosov}
\date{\today}
\address{Software Developer, DevOps Engineer}
\email{kolosovp94@gmail.com}
\urladdr{https://kolosovpetro.github.io}
\keywords{
    Goldbach conjecture,
    Goldbach pairs,
    Minimal Goldbach pairs,
    Primes,
    Twin primes
}
\subjclass[2010]{11N05, 11P32}
\hypersetup{
    pdftitle={Counting primes and twin-primes using minimal Goldbach pairs},
    pdfsubject={
        Goldbach conjecture,
        Goldbach pairs,
        Minimal Goldbach pairs,
        Primes,
        Twin primes
    },
    pdfauthor={Petro Kolosov},
    pdfkeywords={
        Goldbach conjecture,
        Goldbach pairs,
        Minimal Goldbach pairs,
        Primes,
        Twin primes
    }
}
\begin{document}
    \begin{abstract}
        Goldbach's Conjecture assumes that every even integer $2N \geq 4$ can be written as the sum of two primes
$2N = p_i + p_j$, where $(p_i, p_j)$ is called a Goldbach pair.
The minimal Goldbach pair is a pair $(p_i, p_j)$ such that $p_i$ is minimal and $p_j = 2N - p_i$ is also a prime.
We define a function $F_{2N}(P)$ that counts the occurrences of $p_i = P$ in a set of minimal Goldbach pairs
up to $2N$, where $P$ is a fixed prime number.
In particular, the function $F_{2N}(P)$ provides the following identities in terms of prime counting $\pi(2N)$ and
twin-prime counting $\pi_2(2N), \; \pi_4(2N)$
\begin{align*}
    \pi(2N)   &= F_{2N+3}(3) + 1 \\
    \pi_2(2N) &= F_{2N+3}(3) - F_{2N+5}(5) \\
    \pi_4(2N) &= F_{2N}(5) - F_{2N}(7)
\end{align*}

    \end{abstract}

    \maketitle

    \tableofcontents

    \freefootnote{Sources: \url{https://github.com/kolosovpetro/MinimalGoldbachPairsInPrimesCounting}}

    \section{Introduction} \label{sec:introduction}
    This manuscript provides a comprehensive review of the work~\cite{yamagishi2025goldbach} done by Michel Yamagishi,
extending it with additional results.
The Goldbach conjecture asserts that every even integer $2N \geq 4$ is a sum of two primes
\begin{align*}
    2N = p_i + p_j
\end{align*}
where $(p_i, p_j)$ is called a Goldbach pair.

A Goldbach pair is not unique for even integers greater than 6,
which means that there can be multiple Goldbach pairs for even integer $2N \geq 8$.
For example: $10=3+7$ and $10=5+5$ and $10=7+3$, where the Goldbach pairs are $(3,7), \; (5,5), \; (7,3)$.

The minimal Goldbach pair is the pair with the smallest $p_i$ among all goldbach pairs for even integer $2N$.
For the even integer $10$, we have three pairs: $(3,7), \; (5,5), \; (7,3)$
and the minimal one is $(3,7)$ because
$3$ is the smallest among all $p_i$ values: $3, 5, 7$.


Consider the following minimal Goldbach pairs for even integer $2k$ within the range $6 \leq 2k \leq 50$
\begin{align*}
    6  &= 3 + 3,   & 30 &= 7 + 23, \\
    8  &= 3 + 5,   & 32 &= 3 + 29, \\
    10 &= 3 + 7,   & 34 &= 3 + 31, \\
    12 &= 5 + 7,   & 36 &= 5 + 31, \\
    14 &= 3 + 11,  & 38 &= 7 + 31, \\
    16 &= 3 + 13,  & 40 &= 3 + 37, \\
    18 &= 5 + 13,  & 42 &= 5 + 37, \\
    20 &= 3 + 17,  & 44 &= 3 + 41, \\
    22 &= 3 + 19,  & 46 &= 3 + 43, \\
    24 &= 5 + 19,  & 48 &= 5 + 43, \\
    26 &= 3 + 23,  & 50 &= 3 + 47, \\
    28 &= 5 + 23,  &
\end{align*}
We can notice that minimal Goldbach pairs with minimal $p_i=3$ produce a sequence of odd prime numbers
$p_j = 3, \; 5, \; 7, \; 11, \; 13, \; 17 \; \dots$ which is quite remarkable:
\begin{align*}
    6  &= 3 + 3,   & 32 &= 3 + 29, \\
    8  &= 3 + 5,   & 34 &= 3 + 31, \\
    10 &= 3 + 7,   & 40 &= 3 + 37, \\
    14 &= 3 + 11,  & 44 &= 3 + 41, \\
    16 &= 3 + 13,  & 46 &= 3 + 43, \\
    20 &= 3 + 17,  & 50 &= 3 + 47, \\
    22 &= 3 + 19,  &              \\
    26 &= 3 + 23,  &
\end{align*}
Another interesting observation is that by selecting the pairs with minimal $p_i=5$ yields
the sequence of primes $p_j$ such that $p_j+2$ is not a prime
\begin{align*}
    12 &= 5 + 7,   & 36 &= 5 + 31, \\
    18 &= 5 + 13,  & 42 &= 5 + 37, \\
    24 &= 5 + 19,  & 48 &= 5 + 43, \\
    28 &= 5 + 23,  &
\end{align*}
To formalize and clarify our discussion, we define a few functions.
Let $G_{\min} (2N)$ be a function that returns a set of minimal Goldbach pairs $(p_i, p_j)$ having $\min p_i$
over the range $6 \leq 2k \leq 2N$
\begin{align*}
    G_{\min} (2N) = \{(p_i, p_j) \mid p_i + p_j = 2k \mid 6 \leq 2k \leq 2N \mid \min p_i \}
\end{align*}

For example,
\begin{align*}
    G_{\min}(20) = \{
    (3,3),\;
    (3,5),\;
    (3,7),\;
    (5,7),\;
    (3,11),\;
    (3,13),\;
    (5,13),\;
    (3,17)
    \}
\end{align*}
Let $W_{2N}(P)$ be a function that returns the set of elements $p_j$ from $G_{\min} (2N)$ having $p_i=P$
\begin{align*}
    W_{2N}(P) = \{p_j \mid (p_i, p_j) \in G_{\min} (2N) \; \mathrm{and} \; p_i = P \}
\end{align*}
Then, the sequence of odd prime numbers~\cite{oeis:A065091} is given by $W_{2N}(3)$
\begin{align*}
    \{ 3, \; 5, \; 7, \; 11, \; \dots, \; p \leq 2N - 3\} = W_{2N}(3)
\end{align*}
Now we can easily count the number of primes within the interval $6 \leq 2k \leq 2N$ because $\pi(2N)$ is equal to
the total number of elements inside the set $W_{2N}(3)$, which corresponds to the sequence~\cite{oeis_A000720}
\begin{align*}
    \pi(2N) = F_{2N+3}(3) + 1
\end{align*}
where $F_{2N}(3)$ is the function that counts the number of elements inside the set $W_{2N}(3)$.
In general $F_{2N}(P) = |W_{2N}(P)|$.

Taking $P=5$ in $W_{2N}(P)$ yields a sequence of primes $p_j$ such that $p_j+2$ is not a prime~\cite{oeis:A049591}
\begin{align*}
    W_{2N}(5) = \{ 7, \; 13, \; 19, \; 23, \; 31, \; 37, \; 43, \; 47, \; 53, \; \dots, \;  p \leq 2N - 5 \}
\end{align*}
Hence, by excluding the elements of the set $W_{2N}(5)$ from the set $W_{2N}(3)$ yields the sequence of
lesser twin primes~\cite{oeis:A001359}
\begin{align*}
    W_{2N}(3) \setminus W_{2N}(5) = \{ 3, \; 5, \; 11, \; 17, \; 29, \; 41, \; 59, \; 71, \; 101, \; 107, \; 137, \; \dots, \;  p \leq 2N - 3 \}
\end{align*}
This implies that the number of twin primes in range $6 \leq 2k \leq 2N$ can be expressed in terms of $F_{2N}(P)$
\begin{align*}
    \pi_2 (2N) = F_{2N+3}(3) - F_{2N+5}(5)
\end{align*}
For example,
\begin{align*}
    \pi_2 (10) &= F_{10+3}(3) - F_{10+5}(5) = 2 \\
    \pi_2 (100) &= F_{100+3}(3) - F_{100+5}(5) = 8 \\
    \pi_2 (1000) &= F_{1000+3}(3) - F_{1000+5}(5) = 35 \\
    \pi_2 (10000) &= F_{10000+3}(3) - F_{10000+5}(5) = 205 \\
    \pi_2 (100000) &= F_{100000+3}(3) - F_{100000+5}(5) = 1224 \\
    \pi_2 (1000000) &= F_{1000000+3}(3) - F_{1000000+5}(5) = 8169
\end{align*}
These results match the sequence~\cite{oeis_A007508}.

In addition, the functions $W_{2N}(P)$ and $F_{2N}(P)$ provide a way to count cousin
primes $p$ such that $p+4$ is a prime.
We can observe this by excluding the elements of the set $W_{2N}(7)$ from the set $W_{2N}(5)$ which gives
the sequence of cousin primes~\cite{oeis:A023200}
\begin{align*}
    W_{2N}(5) \setminus W_{2N}(7) = \{ 7, \; 13, \; 19, \; 37, \; 43, \; 67, \; 79, \; 97, \; 103, \; 109, \; 127, \; 163, \; \dots, \;  p \leq 2N - 5 \}
\end{align*}
Therefore, the number of cousin primes up to $2N$ can be calculated as
\begin{align*}
    \pi_4 (2N) = F_{2N}(5) - F_{2N}(7)
\end{align*}
For instance,
\begin{align*}
    \pi_4 (10) &= F_{10}(5) - F_{10}(7) = 0 \\
    \pi_4 (100) &= F_{100}(5) - F_{100}(7) = 7 \\
    \pi_4 (1000) &= F_{1000}(5) - F_{1000}(7) = 40 \\
    \pi_4 (10000) &= F_{10000}(5) - F_{10000}(7) = 202 \\
    \pi_4 (100000) &= F_{100000}(5) - F_{100000}(7) = 1215 \\
    \pi_4 (1000000) &= F_{1000000}(5) - F_{1000000}(7) = 8143
\end{align*}
These results match the sequence~\cite{oeis:A093737}.

Having $P=7$ function $W_{2N}(P)$ yields the sequence of primes such that $p_j-p_i \geq 6$,
where $p_j$ is the next prime after $p_i$, see~\cite{oeis:A124582}
\begin{align*}
    W_{2N}(7) = \{ 23, \; 31, \;  47, \; 53, \; 61, \;  73, \; 83, \; 89, \; 113, \; \dots, \; p \leq 2N-7 \}
\end{align*}
This allows us to count the primes for which the next-prime gap at least 6: $\delta_6(2N) = F_{2N+7}(7)$.


    \section{Conclusions}\label{sec:conclusions}
    Assuming Goldbach's Conjecture holds, we introduced a framework based on minimal
Goldbach pairs to derive expressions for key prime-related functions.
Specifically, we defined the function $F_{2N}(P)$ that counts occurrences of primes $p_j$ in minimal Goldbach
pairs $(p_i, p_j)$ where $p_i = P$.
Using this framework, we obtained

\begin{itemize}
    \item The prime-counting function: $\pi(2N) = F_{2N}(3) + 1$
    \item The twin-prime counting function: $\pi_2(2N) = F_{2N}(3) - F_{2N}(5)$
    \item The count of primes with next-prime gap at least 6: $\delta_6(2N) = F_{2N}(7)$
\end{itemize}

These identities establish a novel connection between Goldbach partitions and classical prime number theory.
Computational examples confirm alignment with known integer sequences, reinforcing the potential
of this approach for analytical and numerical exploration of prime distributions.
All the results validated up to $N=10^8$ with source code available on
GitHub~\cite{kolosovpetro2025minimalgoldbachpairs}.



    \bibliographystyle{unsrt}
    \bibliography{MinimalGoldbachPairsInPrimesCounting}
    \noindent \textbf{Version:} \texttt{Local-0.1.0}


\end{document}
